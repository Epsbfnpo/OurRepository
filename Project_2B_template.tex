\documentclass[12pt, a4paper]{article}
\usepackage{graphicx}
\usepackage{color}
\hyphenpenalty=5000
\tolerance=1000

\begin{document}
	\begin{titlepage}
		\begin{center}

			\begin{center}
				\begin{large}
					\begin{tabular}{cc}

					\end{tabular}
				\end{large}
			\end{center}
			
			\textsc{\large Info1111 RE07-01}

			
		\end{center}
	\end{titlepage}
		
	\clearpage

	\tableofcontents

	\clearpage
	\setcounter{page}{1}
	

	\section{Introduction}

	\clearpage
	
	\section{Major Allocation}

	\clearpage
	
	\section{Recommendations}
	\subsection{Computer Science}

	\clearpage
	
	\subsection{Data Science}

	\clearpage
	
	\subsection{Information Systems}

	\subsubsection {INFS 3040}

	\begin{figure}[h]
	\centering
	\includegraphics[width=8cm,height=5cm]{Enterprise_System}
	\caption{Enterprise System}
	\end{figure}

	\paragraph {Brief Introduction}

Nowadays, enterprises spring up like mushrooms in cosmopolis. How they operate? What alteration could they bring to the world? These problems which have long puzzled nonprofessional students will be tackled in this unit. We will have an in-depth understanding of the way in which implementation and use of large scale integrated Enterprise Systems change the nature of organisational capabilities, processes, and roles. Moreover, we will gain considerable hands-on experience with an enterprise-wide system, such as SAP, concentrating on the way in which such systems support integrated business processes. Through a combination of discussion and practical work, we will gain a strong knowledge in both the organisational and technical aspects of Enterprise Systems.

	\paragraph {Which field it pertains to?}

This elective is from an academic unit named ‘Business Information Systems' according to USYD’s standard of classification. Generally, it is supposed to be classified in the field of Commerce and Business.

	\paragraph {What core knowledge it has?} 

In this unit, we can acquire some knowledge and technology below.
 
	\centerline {\textcolor{red}{Basic Introduction}}

	\centerline {\textcolor{red}{Financial accounting}}

This is the most basic part in this unit, mainly introducing the structure and form. Financial accounting is the basic content of any economic or business course. We can learn monetary capital, financial assets, current liabilities and other vital definitions in a gesture to further study in this unit.
 
	\centerline {\textcolor{red}{Procurement}}

	\centerline {\textcolor{red}{Fulfilment}}

This is the chapter of fundamental economic phenomena, containing procurement\footnote{the process of finding and agreeing to terms, and acquiring goods, services, or works from an external source, often via a tendering or competitive bidding process} and fulfilment. We will deeply study two most fundamental economic phenomena (procurement and fulfilment) and how to perform great in these two procedures (dealing methods, methodology and sample cases analyses). 

	\centerline {\textcolor{red}{Software Vendor Evaluation}}

	\centerline {\textcolor{red}{Enterprise Systems and Information Infrastructures}}

This is the chapter of enterprise systems. We will first research several sample cases of software vendor, including digging concrete strengths of them and evaluating the business strategies. Then, we come to the core knowledge of this unit, enterprise systems and information infrastructures. We will systematically learn system security, functions of management systems, dominant modules in ERP\footnote{the integrated management of main business processes, often in real time and mediated by software and technology, usually referred to as a category of business management software—typically a suite of integrated applications—that an organization can use to collect, store, manage, and interpret data from many business activities} and HR management system.

	\centerline {\textcolor{red}{ES Trends - Moving to the Cloud}}

	\centerline {\textcolor{red}{Managing ES - Risks and Controls}}

	\centerline {\textcolor{red}{Managing ES Projects}}

This is the chapter of ES (enterprise systems). We will deeply learn trends, risk controls of ES and self-make ES projects. Followed by the definition of last module.

	\paragraph {What abilities can it improve?}
	
Learning outcomes are provided here.
	
	\centerline {\textcolor{red}{Learning Outcomes}}
	
	\begin{itemize}
	  \item Demonstrate your understanding of the integrated nature of Enterprise Systems and their strategic role in providing a platform for improved business operations.
	  \item Demonstrate your understanding of the role of Enterprise Systems in information infrastructures and evaluate the organisational effects of increasing integration and standardisation.
	  \item Demonstrate your understanding of the issues associated with implementing and supporting Enterprise Systems.
	  \item Demonstrate your understanding of the change undergone in organisations during the implementation and use of Enterprise Systems.
	  \item Demonstrate your theoretical understanding of the organisational structures, master data and processes within SAP.
	  \item Demonstrate your hands-on experience in executing transactions and analysing information using SAP.
	  \item Work both independently and as a responsible member of a diverse team, collaborate and communicate in a professional manner with people from diverse backgrounds.
	\end{itemize}
	
	\paragraph {How it helps the major?}

This unit is definitely useful for students who major in Information System, for the unit's core topic is enterprise systems. We can draw lessons from this unit to construct information system on our own. Moreover, there are many similarities between information system and enterprise system. Enterprise system is made of active and dynamic people, while information system is made of static codes. Hence, in some aspects, it is easier to master the static one than the dynamic one. Elements like field control systems are worth studying in information system.

	\paragraph {How it helps the career?}

After graduation, the majority of us will go to an enterprise for a job (solely a little will settle down in government apartments, self-operated companies). Hence, familiar with how enterprises work will enhance our understanding of where we are and what we are going to do. For example, we can save time and energy while cooperation if we are familiar with the structure of an enterprise; we will know the correct place to ask for assistence when trapped. 

	\paragraph {What opportunities it can bring for us in the future?}

In the future, rapidly increasing technology is an inevitable trend. While some technology may well be confronted with dispute, studying this unit offering us an opportunity to demarcate brand new technology and distinguish the potential ones from mortal ones under the guidance of philosophic ideas.

	\clearpage
	
	\subsubsection {INFS 2020}

	\begin{figure}[h]
	\centering
	\includegraphics[width=8cm,height=5cm]{Business_Process}
	\caption{Business Process}
	\end{figure}

	\paragraph {Brief Introduction}

A business process\footnote{a wide range of structured, often chained, activities or tasks conducted by people or equipment to produce a specific service or product for a particular user or consumer} may be mysterious to a nonprofessional person. They may regard it as simple conversation and money trade. However, numerous details in a business process are waiting to be digged out. Besides acquainted with details, we will establish different models to analyze them systematically. Anchored in these models, we will make further improvement in a gesture to approximate to the real cases. After learning this unit, we might not become an expert at business, but are bound to be better when communicating with developers or disputing demand.

	\paragraph {Which field it pertains to?}

This elective is from an academic unit named ‘Business Information Systems' according to USYD’s standard of classification. Generally, it is supposed to be classified in the field of business and commerce.

	\paragraph {What core knowledge it has?} 

In this unit, we can acquire some knowledge and technology below.
 
\centerline {\textcolor{red}{Basic Introduction}}

\centerline {\textcolor{red}{Process identification}}

\centerline {\textcolor{red}{Process discovery}}

This is the most basic part in this unit, mainly introducing the "Discovery" stage of process. Process identification and discovery are the first two parts of business process. 
 
\centerline {\textcolor{red}{Process modelling using BPMN and Signavio}}

\centerline {\textcolor{red}{Process analysis}}

This chapter focuses on modelling and analysis. In the part of modelling, we will learn two business process diagrams (BPMN\footnote{a graphical representation for specifying business processes in a business process model}, Signavio\footnote{a vendor of Business Process Management (BPM) software based in Berlin and Silicon Valley, main product being Signavio Process Manager, a web-based business process modeling tool}), mainly focusing their scope, elements, flows and artifacts. In another part, we will learn the concrete analysis details when confronted with real cases.

\centerline {\textcolor{red}{Process improvement}}

\centerline {\textcolor{red}{Process implementation and monitoring}}

\centerline {\textcolor{red}{BPM in organisations}}

This chapter focuses on the improvement. Well-established models can be utilized in this part. We will learn to tackle follow-up work after models exist, icluding implementation and monitoring. Moreover, some cases of "BPM in organisations" are provided for us to view and emulate.

	\paragraph {What abilities it can improve?}

	\begin{itemize}
	  \item Discuss the role of Business Process Management (BPM) in a business environment and how BPM initiatives contribute to improving the overall business performance.
	  \item Apply the BPM principles, concepts, and frameworks to business problems and practice as captured in the BPM lifecycle.
	  \item Address the issues and challenges associated with process improvement initiatives in organisations.
	  \item Evaluate the impact of process improvement decisions on stakeholders and the role of change management in addressing the risks associated with such changes in organisations.
	  \item Use BPM tools to model, document, and analyse processes during the BPM lifecycle stages.
	  \item Work independently and collaborate in a professional manner with people from diverse backgrounds within a team work environment.
	\end{itemize}

	\paragraph {How it helps the major?}

This unit set up a complete and clear process in the field of business. However, we can emulate this pattern in Information System. For instance, when we start to hold an information system, we can always make identification and discovery at first. Then, we need to model and analyze a concrete system, containing codes and algorithm. Ultimately, make improvement to the system and conduct things within ethics.

	\paragraph {How it helps the career?}

Familiar with business process are contributing to the further way in enterprises (Very similar with the point of view in last subsection)

	\paragraph {What opportunities it can bring for us in the future?}

In the future, we may accept some high-demanding requests about constructing or modifying information systems. We are bound to have business with the first parties. At this time, familiar with business process will enable us to find win-win methods more easily and confirm mutual goals more rapidly, which shows advantages over our slow-moving developers.

	\clearpage
	
	\subsection{Software Development}
	

	\clearpage

	\section{Contributions}

	\clearpage
	

	\bibliography{refs}
	
	
\end{document} 
