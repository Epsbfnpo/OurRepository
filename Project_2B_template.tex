\documentclass[12pt, a4paper]{article}
\usepackage{graphicx}
\usepackage{color}
\hyphenpenalty=5000
\tolerance=1000

\begin{document}
	\begin{titlepage}
		\begin{center}

			\begin{center}
				\begin{large}
					\begin{tabular}{cc}

					\end{tabular}
				\end{large}
			\end{center}
			
			\textsc{\large Info1111 RE07-01}

			
		\end{center}
	\end{titlepage}
		
	\clearpage

	\tableofcontents

	\clearpage
	\setcounter{page}{1}
	

	\section{Introduction}

	\clearpage
	
	\section{Major Allocation}

	\clearpage
	
	\section{Recommendations}
	\subsection{Computer Science}

	\subsubsection {AREC 3002}

	\begin{figure}[h]
	\centering
	\includegraphics[width=8cm,height=5cm]{Agricultural_Marketing}
	\caption{Agricultural Marketing}
	\end{figure}

	\paragraph {Brief Introduction}

Agricultural market is an important economical market for human beings that exists for many years. This market help farmers sell their crops to people so that people are able to gain enough food. It is one of the earliest markets for people to develop information exchange, because farmers use the market to know how great demand in some agricultural products has potential by observing the agricultural trading in the market. As time goes by, computer science grows rapidly today. The knowledge of computer science could cause a big impact on agricultural market. It is good for us to combine a mature market and new skills together which help develop the modern agricultural market today.

	\paragraph {Which field it pertains to?}

The elective is from an academic unit named Environmental, Agricultural and Resource Economics according to USYD’s standard of classification. The elective is classified as the field of economics and marketing.

	\paragraph {What core knowledge it has?}

The purpose of the elective is to provide an understanding of the fundamental force leading the agricultural market. It talks about the analysis of price and different parts of form such as time, space, technology in this market. There is some knowledge about marketing like contracts and consumer concerns, risk-sharing devices, and information. The content of the unit of study is analytical because the agricultural market is built on microeconomic theory on its application which is the production and consumption in the market.

In the textbook, Agricultural Product Price, written by William G. Tomek and Harry M. Kaiser\footnote{an economist and the Gellert Family Professor of Applied Economics and Management at Cornell University, one of the first economists to investigate the economic impacts of climate change on agriculture}, divided into three parts discussing the agricultural market.

	\centerline {\textcolor{red}{Principles of Price Determination(P93)}}

It includes the content of basic understanding of theoretical concepts of determination on agricultural price. It illustrates the special relationship between demand and supply in agricultural commodities.

	\centerline {\textcolor{red}{Price Differences and Variability(P115)}}

It drives readers to the aspect of difference in price because of the many levels of markets, and the relationship between general farm price and non-farm price in product attributes.

	\centerline {\textcolor{red}{Pricing Institutions(P81)}}

For the third part it points out the pricing ranging for agricultural commodities. And some implication of price in agricultural commodities from government policy.

	\paragraph {What abilities can it improve?}
	
Learning outcomes and graduate qualities are provided here.
	
	\centerline {\textcolor{red}{Learning Outcomes}}
	
	\begin{itemize}
	  \item Demonstrate an understanding of the fundamental concepts in the theory of agricultural markets, and their application in the context of domestic and international agribusiness.
	  \item Demonstrate analytical abilities in the context of agricultural markets.
	\end{itemize}

	\centerline {\textcolor{red}{Graduate Qualities}}
	
	\begin{itemize}
	  \item Depth of disciplinary expertise
	  \item Critical thinking and problem solving
	  \item Oral and written communication
	  \item Information and digital literacy
	  \item Inventiveness
	  \item Cultural competence
	  \item Interdisciplinary effectiveness
	  \item Integrated professional, ethical, and personal identity
	  \item Influence
	\end{itemize}

	\paragraph {How it helps the major?}

This unit of study is not for studying computer science, but the market is formed by humans and products which are the important and original source for data building. The unique model of agricultural marketing can bring ideas to those computer scientists when they are creating Algorithms and data structures. Students would have a good understanding of learning computer science if its knowledge is applied to the computing project of
economics.

	\paragraph {How it helps the career?}

If I start my career as a database administrator who works in an agricultural industry, I can collect any information such as the number of animals, milk production, reproduction of crops from the agricultural market. The knowledge of marketing and the skills of computing lead me to form a systematic business model for producing commodities in the market. The amount of irrigation could be controlled and make assumptions for the plant growth.

	\paragraph {What opportunities it can bring for us in the future?}

In the future, with increasing demand for agricultural products as the number of people grows rapidly right now, it will be a huge market in agricultural areas. The market will require many high-tech people in the future. The next level of upgrade for the agricultural market is unstoppable when skilled scientists conduct research and make some changes in the market.

	\clearpage

	\subsubsection {EXSS 3040}

	\begin{figure}[h]
	\centering
	\includegraphics[width=8cm,height=5cm]{Physiological_Testing}
	\caption{Physiological Testing}
	\end{figure}

	\paragraph {Brief Introduction}

Everyone would like to do more exercise to make their bodies better. How should trainers know whether their ways of training help improve their bodies effectively? Physiological testing and training is the path of finding yourself in a scientific way. The study of physiological assessment for athletes, and scientific training is a significant part of the subject of movement science. Scientific ways of training and testing could help athletes improve rapidly.  The analysis of those sporting data collected from the testing and training require computing languages to organize.
	
	\paragraph {Which field it pertains to?}

The elective is from an academic unit named Movement Sciences according to UYSD’s standard of classification. The elective is classified as the field of Exercise Physiology, applied science. 

	\paragraph {What core knowledge it has?}

Physiological testing and training give students theoretical concepts and practical ability to the application of testing and measurement in sport science. The unit of study emphasizes the role of speed, strength, endurance, and agility in the sports area. There are some lectures, case studies, laboratories for investigating physiological nature.

	\centerline {\textcolor{red}{Testing Muscular Strength and Power}}

It talks about the golden standard of testing skeletal muscle power from movement. It is called vertical leap. There are some methods which also evaluate strength and power of the muscle. It also includes an activity with a clear instruction on how to measure strength and power.

	\centerline {\textcolor{red}{Testing and Training Speed, Agility and Change of Direction}}

It introduces the implication of training. It discusses the relationship between agility tests and COD (change of direction) tests. Some studies conclude that strength qualities could transfer to gain in CODS.

	\paragraph {What abilities can it improve?}
	
Learning outcomes are provided here.
	
	\centerline {\textcolor{red}{Learning Outcomes}}
	
	\begin{itemize}
	  \item Have a good understanding of physiological testing and training.
	  \item Demonstrates on how to use measurement skills to train and test athletes.
	  \item Provide sporting advice on human bodies according to training theory and practice
	\end{itemize}

	\paragraph {How it helps the major?}

The unit of study is to explore movement science and human bodies. In computer science, there are so many branches that are related to human technology. Physiological testing and training provide significant information to the computation system as one of the categories in the database. The experiences and knowledge of learning testing are useful for me to do the coding test during each project in computer science. The unit of study shares the same trait with computer science, they both list plans of what to do for the specific project and then follow the command and do the testing. Therefore, the logistic theory of this elective works for computer science.

	\paragraph {How it helps the career?}

This elective makes me familiar with human structure and human movement. If I start my IT career in the future, that information and knowledge will help me concentrate on robotic and sports development. I will use my computing skill to produce high-tech products according to the theory of psychological knowledge from testing and training.

	\paragraph {What opportunities it can bring for us in the future?}

In the future, as many people pay attention to their health, it requires a systemic network that helps them improve their bodies for exercise. The testing and training on humans bring valuable data for developing sports science.

	\clearpage
	
	\subsection{Data Science}

	\subsubsection {ENGG 1813}

	\begin{figure}[h]
	\centering
	\includegraphics[width=8cm,height=5cm]{Engineering_Critical_Thinking}
	\caption{Engineering Critical Thinking}
	\end{figure}

	\paragraph {Brief Introduction}

According to the guideline of ENGG1813 Engineering Critical Thinking, it is a unit focus on the rigors of communication in an engineering context including technical writing, teamwork, formal presentations and critical analysis. Critical thinking is a compound thinking tool that people use logic and its related skills to make secondary evaluation of a series of thinking, such as viewpoint, judgment, proposition, argument, etc. At present, the lack of critical thinking in engineering education is related to the shallow problem consciousness, weak professional interest and weak practical ability in engineering talent training. Therefore, it is necessary for engineering students to learn Engineering critical thinking.

	\paragraph {Which field it pertains to?}

The elective (ENGG1813 Engineering Critical Thinking) is from the Faculty of Engineering to USYD’s standard of classifification. Briefly, it is under the table s electives- Engineering.

	\paragraph {What core knowledge it has?}

By studying this unit, we can master various knowledge mentioned below.

Critical thinking is the inherent requirement of cultivating engineering innovative talents. Critical thinking, which refers to the logical thinking ability that can grasp the main points, be good at questioning, discriminating, strictly inferring, rich in wit, clear and agile. The essence of the difference of people's quality is not the difference of their knowledge and information, but the difference of their thinking ability. The critical thinking of cultivating engineering talents is to explore and practice effective ways to reveal and train the differences of this ability, and cultivate them into high-quality innovative talents.

Engineering innovation should complete the knowledge inheritance, and on this basis, be able to carry out engineering innovation and solve new problems in engineering practice.

Engineering students can not only solve new problems, but also carry out process innovation in engineering creation which is related to the critical thinking.

	\paragraph {What abilities can it improve?}
	
Learning outcomes are provided here.
	
	\centerline {\textcolor{red}{Learning Outcomes}}
	
	\begin{itemize}
	  \item Practicing critical thinking and Intensive English language (in an engineering background).
	  \item Intensive focus on the language of Engineering in English.
	  \item Developing cross-cultural awareness.
	  \item Increasing critical thinking capacity.
	\end{itemize}
	
	\centerline {\textcolor{red}{Extra Explanation}}

For LO1, the unit concentrates to develop students’ professional skills and confidence in order to give they better engagement in the study period at the University of Sydney. With lessons about the engineering critical thinking and exercises of writing in team, students from all faculty will have more passion for engineering and their comprehend of the unit will also be enriched. In addition, the capability of students to overwhelm the challenges and frustrations discover on their way of studying.

For LO2, it is very important to master some English professional terms of engineering, so that we can learn other engineering related courses more efficiently.

For LO3, consolidating the important facets and exercises in the tradition science and research-based technical ability of appropriate reasoning.

For LO4, Let students equip with engineering critical thinking skills have a preparation for further study under the Engineering background. 

	\paragraph {How it helps the major?}

This unit is suitable for students who major in data science because it can assist the study of data science. When students majoring in data science look for errors in experiments, it is easy to leave out details, such as selection bias, observer bias and etc. Besides, data science students should beware of clean data syndrom and not give into the fallacies with the use of critical thinking. Under such circumstances, mastering a skill of engineering critical thinking is essential.

	\paragraph {How it helps the career?}

It is obvious that ENGG1813 Engineering Critical Thinking is useful for our data science students in various senses. For the future career, students majoring in data science can engaged in big data management, research and application development in government agencies, enterprises and companies. At the same time, we can get postgraduate students majoring in software engineering, computer science and technology, and applied statistics or go abroad for further study. Generally, graduates step into those jobs related to the data science major will have more promotion space, and as a compound talent, they will get better development, and their business ability will be better.

	\paragraph {What opportunities it can bring for us in the future?}

In the future, the trend of multidisciplinary crossing is inevitable. Data Science is merely a powerful tool, or a compelling concept. There must be an appropriate object to use it, and engineering is the special one. Demand for big data and accurate measurement is increasing, thereforing learning this unit is bound to obtain work opportunities for data scientists.

	\clearpage

	\subsubsection {ECON 1001}

	\begin{figure}[h]
	\centering
	\includegraphics[width=8cm,height=5cm]{Introductory_Microeconomics}
	\caption{Introductory Microeconomics}
	\end{figure}

	\paragraph {Brief Introduction}

According to the guideline of ECON1001 Introductory Microeconomics, it is a unit of introduction to Microeconomics related to the economic decisions based on individual company and households and how they interact in markets with each other. It is common knowledge that economic problems have frequent impact on Australian society. Introductory Microeconomics lead students to explore the language and analytical framework adopted in Economics for the  social phenomenon and public policy issues. Whatever one's career intentions, coming to grips with economic ideas is essential for understanding society, business and government. Students are given a comprehensive introduction to these ideas and are prepared for the advanced study of microeconomics in subsequent years. Prior knowledge of mathematics is assumed.
	
	\paragraph {Which field it pertains to?}

The elective (ECON1001 Introductory Microeconomics) is from the Faculty of Arts and Social Sciences according to the classification of University of Sydney standard. Briefly, it is under the table s electives- Economy.

	\paragraph {What core knowledge it has?}

By studying this unit, we can master various knowledge mentioned below.
Microeconomics is an economic theory that studies the economic behavior of a single economic unit in a society and how the single value of the corresponding economic variable is determined, and analyzes the economic behavior of an individual economic unit.

	\paragraph {What abilities can it improve?}
	
Learning outcomes are provided here.
	
	\centerline {\textcolor{red}{Learning Outcomes}}
	
	\begin{itemize}
	  \item Application the economic concepts into practice.
	  \item Critically estimate the hypothesis or real issues.
	\end{itemize}

	\centerline {\textcolor{red}{Extra Explanation}}

For LO1, after understanding the equilibrium price theory, consumer behavior theory, producer behavior theory (including production theory, cost theory and market equilibrium theory), distribution theory, general equilibrium theory and welfare economics, market failure and microeconomics policy, we can apply them into our real life problems to deal with decision making.

For LO2, when we meet with problems, with the knowledge of microeconomics, we can take the limitation and benefit both into consideration to find out a solution.

	\paragraph {How it helps the major?}

The knowledge in this unit can be certain useful tool for a students who major in data science because it can assist the study of data science. For example, microeconomics can help data science learner understand consumers' decisions and enterprises make strategic decisions (supply chain) according to consumers' react.

	\paragraph {How it helps the career?}

As is said in the unit outlines of the unit, whatever one's career intentions, coming to grips with economic ideas is essential for understanding society, business and government. 

In terms of future career path, Econ1001 can also be helpful, broadening students’ horizons and providing them with more options and opportunities to select the most suitable jobs. I have mentioned previously that learning microeconomics enables students to see how data science can be utilized for business models and assist precise decision making. Taking microeconomics can also help students to take a peek into the financial and economics disciplines. Considering the amount of data science and data analytics needed in the field of financial market and economics researches, taking microeconomics may be helpful for students to discover a new career interest and have more choices. Data science as an academic discipline is far more than just the knowledge of data. It has great implications in a broad range of career paths, and for students majoring in data science to be successful, they need to know not only data science, but also have a comprehensive understanding in the field that they utilize their data science expertise in. Therefore, microeconomics can be a very sound choice for more opportunities and a broader career path.

	\paragraph {What opportunities it can bring for us in the future?}

An economic course can provide great opportunities about multidisciplinary fields like EC (electronic commerce). Also, Yale University set a combined subject concerning computing and economy in 2019, which means the relationship between these two subjects are no more alienated.

	\clearpage
	
	\subsection{Information Systems}

	\subsubsection {INFS 3040}

	\begin{figure}[h]
	\centering
	\includegraphics[width=8cm,height=5cm]{Enterprise_System}
	\caption{Enterprise System}
	\end{figure}

	\paragraph {Brief Introduction}

Nowadays, enterprises spring up like mushrooms in cosmopolis. How they operate? What alteration could they bring to the world? These problems which have long puzzled nonprofessional students will be tackled in this unit. We will have an in-depth understanding of the way in which implementation and use of large scale integrated Enterprise Systems change the nature of organisational capabilities, processes, and roles. Moreover, we will gain considerable hands-on experience with an enterprise-wide system, such as SAP, concentrating on the way in which such systems support integrated business processes. Through a combination of discussion and practical work, we will gain a strong knowledge in both the organisational and technical aspects of Enterprise Systems.

	\paragraph {Which field it pertains to?}

This elective is from an academic unit named ‘Business Information Systems' according to USYD’s standard of classification. Generally, it is supposed to be classified in the field of Commerce and Business.

	\paragraph {What core knowledge it has?} 

In this unit, we can acquire some knowledge and technology below.
 
	\centerline {\textcolor{red}{Basic Introduction}}

	\centerline {\textcolor{red}{Financial accounting}}

This is the most basic part in this unit, mainly introducing the structure and form. Financial accounting is the basic content of any economic or business course. We can learn monetary capital, financial assets, current liabilities and other vital definitions in a gesture to further study in this unit.
 
	\centerline {\textcolor{red}{Procurement}}

	\centerline {\textcolor{red}{Fulfilment}}

This is the chapter of fundamental economic phenomena, containing procurement\footnote{the process of finding and agreeing to terms, and acquiring goods, services, or works from an external source, often via a tendering or competitive bidding process} and fulfilment. We will deeply study two most fundamental economic phenomena (procurement and fulfilment) and how to perform great in these two procedures (dealing methods, methodology and sample cases analyses). 

	\centerline {\textcolor{red}{Software Vendor Evaluation}}

	\centerline {\textcolor{red}{Enterprise Systems and Information Infrastructures}}

This is the chapter of enterprise systems. We will first research several sample cases of software vendor, including digging concrete strengths of them and evaluating the business strategies. Then, we come to the core knowledge of this unit, enterprise systems and information infrastructures. We will systematically learn system security, functions of management systems, dominant modules in ERP\footnote{the integrated management of main business processes, often in real time and mediated by software and technology, usually referred to as a category of business management software—typically a suite of integrated applications—that an organization can use to collect, store, manage, and interpret data from many business activities} and HR management system.

	\centerline {\textcolor{red}{ES Trends - Moving to the Cloud}}

	\centerline {\textcolor{red}{Managing ES - Risks and Controls}}

	\centerline {\textcolor{red}{Managing ES Projects}}

This is the chapter of ES (enterprise systems). We will deeply learn trends, risk controls of ES and self-make ES projects. Followed by the definition of last module.

	\paragraph {What abilities can it improve?}
	
Learning outcomes are provided here.
	
	\centerline {\textcolor{red}{Learning Outcomes}}
	
	\begin{itemize}
	  \item Demonstrate your understanding of the integrated nature of Enterprise Systems and their strategic role in providing a platform for improved business operations.
	  \item Demonstrate your understanding of the role of Enterprise Systems in information infrastructures and evaluate the organisational effects of increasing integration and standardisation.
	  \item Demonstrate your understanding of the issues associated with implementing and supporting Enterprise Systems.
	  \item Demonstrate your understanding of the change undergone in organisations during the implementation and use of Enterprise Systems.
	  \item Demonstrate your theoretical understanding of the organisational structures, master data and processes within SAP.
	  \item Demonstrate your hands-on experience in executing transactions and analysing information using SAP.
	  \item Work both independently and as a responsible member of a diverse team, collaborate and communicate in a professional manner with people from diverse backgrounds.
	\end{itemize}
	
	\paragraph {How it helps the major?}

This unit is definitely useful for students who major in Information System, for the unit's core topic is enterprise systems. We can draw lessons from this unit to construct information system on our own. Moreover, there are many similarities between information system and enterprise system. Enterprise system is made of active and dynamic people, while information system is made of static codes. Hence, in some aspects, it is easier to master the static one than the dynamic one. Elements like field control systems are worth studying in information system.

	\paragraph {How it helps the career?}

After graduation, the majority of us will go to an enterprise for a job (solely a little will settle down in government apartments, self-operated companies). Hence, familiar with how enterprises work will enhance our understanding of where we are and what we are going to do. For example, we can save time and energy while cooperation if we are familiar with the structure of an enterprise; we will know the correct place to ask for assistence when trapped. 

	\paragraph {What opportunities it can bring for us in the future?}

In the future, rapidly increasing technology is an inevitable trend. While some technology may well be confronted with dispute, studying this unit offering us an opportunity to demarcate brand new technology and distinguish the potential ones from mortal ones under the guidance of philosophic ideas.

	\clearpage
	
	\subsubsection {INFS 2020}

	\begin{figure}[h]
	\centering
	\includegraphics[width=8cm,height=5cm]{Business_Process}
	\caption{Business Process}
	\end{figure}

	\paragraph {Brief Introduction}

A business process\footnote{a wide range of structured, often chained, activities or tasks conducted by people or equipment to produce a specific service or product for a particular user or consumer} may be mysterious to a nonprofessional person. They may regard it as simple conversation and money trade. However, numerous details in a business process are waiting to be digged out. Besides acquainted with details, we will establish different models to analyze them systematically. Anchored in these models, we will make further improvement in a gesture to approximate to the real cases. After learning this unit, we might not become an expert at business, but are bound to be better when communicating with developers or disputing demand.

	\paragraph {Which field it pertains to?}

This elective is from an academic unit named ‘Business Information Systems' according to USYD’s standard of classification. Generally, it is supposed to be classified in the field of business and commerce.

	\paragraph {What core knowledge it has?} 

In this unit, we can acquire some knowledge and technology below.
 
\centerline {\textcolor{red}{Basic Introduction}}

\centerline {\textcolor{red}{Process identification}}

\centerline {\textcolor{red}{Process discovery}}

This is the most basic part in this unit, mainly introducing the "Discovery" stage of process. Process identification and discovery are the first two parts of business process. 
 
\centerline {\textcolor{red}{Process modelling using BPMN and Signavio}}

\centerline {\textcolor{red}{Process analysis}}

This chapter focuses on modelling and analysis. In the part of modelling, we will learn two business process diagrams (BPMN\footnote{a graphical representation for specifying business processes in a business process model}, Signavio\footnote{a vendor of Business Process Management (BPM) software based in Berlin and Silicon Valley, main product being Signavio Process Manager, a web-based business process modeling tool}), mainly focusing their scope, elements, flows and artifacts. In another part, we will learn the concrete analysis details when confronted with real cases.

\centerline {\textcolor{red}{Process improvement}}

\centerline {\textcolor{red}{Process implementation and monitoring}}

\centerline {\textcolor{red}{BPM in organisations}}

This chapter focuses on the improvement. Well-established models can be utilized in this part. We will learn to tackle follow-up work after models exist, icluding implementation and monitoring. Moreover, some cases of "BPM in organisations" are provided for us to view and emulate.

	\paragraph {What abilities it can improve?}

	\begin{itemize}
	  \item Discuss the role of Business Process Management (BPM) in a business environment and how BPM initiatives contribute to improving the overall business performance.
	  \item Apply the BPM principles, concepts, and frameworks to business problems and practice as captured in the BPM lifecycle.
	  \item Address the issues and challenges associated with process improvement initiatives in organisations.
	  \item Evaluate the impact of process improvement decisions on stakeholders and the role of change management in addressing the risks associated with such changes in organisations.
	  \item Use BPM tools to model, document, and analyse processes during the BPM lifecycle stages.
	  \item Work independently and collaborate in a professional manner with people from diverse backgrounds within a team work environment.
	\end{itemize}

	\paragraph {How it helps the major?}

This unit set up a complete and clear process in the field of business. However, we can emulate this pattern in Information System. For instance, when we start to hold an information system, we can always make identification and discovery at first. Then, we need to model and analyze a concrete system, containing codes and algorithm. Ultimately, make improvement to the system and conduct things within ethics.

	\paragraph {How it helps the career?}

Familiar with business process are contributing to the further way in enterprises (Very similar with the point of view in last subsection)

	\paragraph {What opportunities it can bring for us in the future?}

In the future, we may accept some high-demanding requests about constructing or modifying information systems. We are bound to have business with the first parties. At this time, familiar with business process will enable us to find win-win methods more easily and confirm mutual goals more rapidly, which shows advantages over our slow-moving developers.

	\clearpage
	
	\subsection{Software Development}
	

	\clearpage

	\section{Contributions}

	\clearpage
	

	\bibliography{refs}
	
	
\end{document}
