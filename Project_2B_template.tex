\documentclass[12pt, a4paper]{article}
\usepackage{graphicx}
\usepackage{color}
\hyphenpenalty=5000
\tolerance=1000

\begin{document}
	\begin{titlepage}
		\begin{center}

			\begin{center}
				\begin{large}
					\begin{tabular}{cc}

					\end{tabular}
				\end{large}
			\end{center}
			
			\textsc{\large Info1111 RE07-01}

			
		\end{center}
	\end{titlepage}
		
	\clearpage

	\tableofcontents

	\clearpage
	\setcounter{page}{1}
	

	\section{Introduction}

	\clearpage
	
	\section{Major Allocation}

	\clearpage
	
	\section{Recommendations}
	\subsection{Computer Science}

	\subsubsection {AREC 3002}

	\begin{figure}[h]
	\centering
	\includegraphics[width=8cm,height=5cm]{Agricultural_Marketing}
	\caption{Agricultural Marketing}
	\end{figure}

	\paragraph {Brief Introduction}

Agricultural market is an important economical market for human beings that exists for many years. This market help farmers sell their crops to people so that people are able to gain enough food. It is one of the earliest markets for people to develop information exchange, because farmers use the market to know how great demand in some agricultural products has potential by observing the agricultural trading in the market. As time goes by, computer science grows rapidly today. The knowledge of computer science could cause a big impact on agricultural market. It is good for us to combine a mature market and new skills together which help develop the modern agricultural market today.

	\paragraph {Which field it pertains to?}

The elective is from an academic unit named Environmental, Agricultural and Resource Economics according to USYD’s standard of classification. The elective is classified as the field of economics and marketing.

	\paragraph {What core knowledge it has?}

The purpose of the elective is to provide an understanding of the fundamental force leading the agricultural market. It talks about the analysis of price and different parts of form such as time, space, technology in this market. There is some knowledge about marketing like contracts and consumer concerns, risk-sharing devices, and information. The content of the unit of study is analytical because the agricultural market is built on microeconomic theory on its application which is the production and consumption in the market.

In the textbook, Agricultural Product Price, written by William G. Tomek and Harry M. Kaiser\footnote{an economist and the Gellert Family Professor of Applied Economics and Management at Cornell University, one of the first economists to investigate the economic impacts of climate change on agriculture}, divided into three parts discussing the agricultural market.

	\centerline {\textcolor{red}{Principles of Price Determination(P93)}}

It includes the content of basic understanding of theoretical concepts of determination on agricultural price. It illustrates the special relationship between demand and supply in agricultural commodities.

	\centerline {\textcolor{red}{Price Differences and Variability(P115)}}

It drives readers to the aspect of difference in price because of the many levels of markets, and the relationship between general farm price and non-farm price in product attributes.

	\centerline {\textcolor{red}{Pricing Institutions(P81)}}

For the third part it points out the pricing ranging for agricultural commodities. And some implication of price in agricultural commodities from government policy.

	\paragraph {What abilities can it improve?}
	
Learning outcomes and graduate qualities are provided here.
	
	\centerline {\textcolor{red}{Learning Outcomes}}
	
	\begin{itemize}
	  \item Demonstrate an understanding of the fundamental concepts in the theory of agricultural markets, and their application in the context of domestic and international agribusiness.
	  \item Demonstrate analytical abilities in the context of agricultural markets.
	\end{itemize}

	\centerline {\textcolor{red}{Graduate Qualities}}
	
	\begin{itemize}
	  \item Depth of disciplinary expertise
	  \item Critical thinking and problem solving
	  \item Oral and written communication
	  \item Information and digital literacy
	  \item Inventiveness
	  \item Cultural competence
	  \item Interdisciplinary effectiveness
	  \item Integrated professional, ethical, and personal identity
	  \item Influence
	\end{itemize}

	\paragraph {How it helps the major?}

This unit of study is not for studying computer science, but the market is formed by humans and products which are the important and original source for data building. The unique model of agricultural marketing can bring ideas to those computer scientists when they are creating Algorithms and data structures. Students would have a good understanding of learning computer science if its knowledge is applied to the computing project of
economics.

	\paragraph {How it helps the career?}

If I start my career as a database administrator who works in an agricultural industry, I can collect any information such as the number of animals, milk production, reproduction of crops from the agricultural market. The knowledge of marketing and the skills of computing lead me to form a systematic business model for producing commodities in the market. The amount of irrigation could be controlled and make assumptions for the plant growth.

	\paragraph {What opportunities it can bring for us in the future?}

In the future, with increasing demand for agricultural products as the number of people grows rapidly right now, it will be a huge market in agricultural areas. The market will require many high-tech people in the future. The next level of upgrade for the agricultural market is unstoppable when skilled scientists conduct research and make some changes in the market.

	\clearpage

	\subsubsection {EXSS 3040}

	\begin{figure}[h]
	\centering
	\includegraphics[width=8cm,height=5cm]{Physiological_Testing}
	\caption{Physiological Testing}
	\end{figure}

	\paragraph {Brief Introduction}

Everyone would like to do more exercise to make their bodies better. How should trainers know whether their ways of training help improve their bodies effectively? Physiological testing and training is the path of finding yourself in a scientific way. The study of physiological assessment for athletes, and scientific training is a significant part of the subject of movement science. Scientific ways of training and testing could help athletes improve rapidly.  The analysis of those sporting data collected from the testing and training require computing languages to organize.
	
	\paragraph {Which field it pertains to?}

The elective is from an academic unit named Movement Sciences according to UYSD’s standard of classification. The elective is classified as the field of Exercise Physiology, applied science. 

	\paragraph {What core knowledge it has?}

Physiological testing and training give students theoretical concepts and practical ability to the application of testing and measurement in sport science. The unit of study emphasizes the role of speed, strength, endurance, and agility in the sports area. There are some lectures, case studies, laboratories for investigating physiological nature.

	\centerline {\textcolor{red}{Testing Muscular Strength and Power}}

It talks about the golden standard of testing skeletal muscle power from movement. It is called vertical leap. There are some methods which also evaluate strength and power of the muscle. It also includes an activity with a clear instruction on how to measure strength and power.

	\centerline {\textcolor{red}{Testing and Training Speed, Agility and Change of Direction}}

It introduces the implication of training. It discusses the relationship between agility tests and COD (change of direction) tests. Some studies conclude that strength qualities could transfer to gain in CODS.

	\paragraph {What abilities can it improve?}
	
Learning outcomes are provided here.
	
	\centerline {\textcolor{red}{Learning Outcomes}}
	
	\begin{itemize}
	  \item Have a good understanding of physiological testing and training.
	  \item Demonstrates on how to use measurement skills to train and test athletes.
	  \item Provide sporting advice on human bodies according to training theory and practice
	\end{itemize}

	\paragraph {How it helps the major?}

The unit of study is to explore movement science and human bodies. In computer science, there are so many branches that are related to human technology. Physiological testing and training provide significant information to the computation system as one of the categories in the database. The experiences and knowledge of learning testing are useful for me to do the coding test during each project in computer science. The unit of study shares the same trait with computer science, they both list plans of what to do for the specific project and then follow the command and do the testing. Therefore, the logistic theory of this elective works for computer science.

	\paragraph {How it helps the career?}

This elective makes me familiar with human structure and human movement. If I start my IT career in the future, that information and knowledge will help me concentrate on robotic and sports development. I will use my computing skill to produce high-tech products according to the theory of psychological knowledge from testing and training.

	\paragraph {What opportunities it can bring for us in the future?}

In the future, as many people pay attention to their health, it requires a systemic network that helps them improve their bodies for exercise. The testing and training on humans bring valuable data for developing sports science.

	\clearpage
	
	\subsection{Data Science}

	\clearpage
	
	\subsection{Information Systems}

	\clearpage
	
	\subsection{Software Development}

	\clearpage

	\section{Contributions}

	\clearpage
	

	\bibliography{refs}
	
	
\end{document} 
