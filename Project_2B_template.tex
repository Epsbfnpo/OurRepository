\documentclass[12pt, a4paper]{article}
\usepackage{graphicx}
\usepackage{color}
\hyphenpenalty=5000
\tolerance=1000

\begin{document}
	\begin{titlepage}
		\begin{center}

			\begin{center}
				\begin{large}
					\begin{tabular}{cc}

					\end{tabular}
				\end{large}
			\end{center}
			
			\textsc{\large Info1111 RE07-01}

			
		\end{center}
	\end{titlepage}
		
	\clearpage

	\tableofcontents

	\clearpage
	\setcounter{page}{1}
	

	\section{Introduction}

	\clearpage
	
	\section{Major Allocation}

	\clearpage
	
	\section{Recommendations}
	\subsection{Computer Science}

	\clearpage
	
	\subsection{Data Science}

	\subsubsection {ENGG 1813}

	\begin{figure}[h]
	\centering
	\includegraphics[width=8cm,height=5cm]{Engineering_Critical_Thinking}
	\caption{Engineering Critical Thinking}
	\end{figure}

	\paragraph {Brief Introduction}

According to the guideline of ENGG1813 Engineering Critical Thinking, it is a unit focus on the rigors of communication in an engineering context including technical writing, teamwork, formal presentations and critical analysis. Critical thinking is a compound thinking tool that people use logic and its related skills to make secondary evaluation of a series of thinking, such as viewpoint, judgment, proposition, argument, etc. At present, the lack of critical thinking in engineering education is related to the shallow problem consciousness, weak professional interest and weak practical ability in engineering talent training. Therefore, it is necessary for engineering students to learn Engineering critical thinking.

	\paragraph {Which field it pertains to?}

The elective (ENGG1813 Engineering Critical Thinking) is from the Faculty of Engineering to USYD’s standard of classifification. Briefly, it is under the table s electives- Engineering.

	\paragraph {What core knowledge it has?}

By studying this unit, we can master various knowledge mentioned below.

Critical thinking is the inherent requirement of cultivating engineering innovative talents. Critical thinking, which refers to the logical thinking ability that can grasp the main points, be good at questioning, discriminating, strictly inferring, rich in wit, clear and agile. The essence of the difference of people's quality is not the difference of their knowledge and information, but the difference of their thinking ability. The critical thinking of cultivating engineering talents is to explore and practice effective ways to reveal and train the differences of this ability, and cultivate them into high-quality innovative talents.

Engineering innovation should complete the knowledge inheritance, and on this basis, be able to carry out engineering innovation and solve new problems in engineering practice.

Engineering students can not only solve new problems, but also carry out process innovation in engineering creation which is related to the critical thinking.

	\paragraph {What abilities can it improve?}
	
Learning outcomes are provided here.
	
	\centerline {\textcolor{red}{Learning Outcomes}}
	
	\begin{itemize}
	  \item Practicing critical thinking and Intensive English language (in an engineering background).
	  \item Intensive focus on the language of Engineering in English.
	  \item Developing cross-cultural awareness.
	  \item Increasing critical thinking capacity.
	\end{itemize}
	
	\centerline {\textcolor{red}{Extra Explanation}}

For LO1, the unit concentrates to develop students’ professional skills and confidence in order to give they better engagement in the study period at the University of Sydney. With lessons about the engineering critical thinking and exercises of writing in team, students from all faculty will have more passion for engineering and their comprehend of the unit will also be enriched. In addition, the capability of students to overwhelm the challenges and frustrations discover on their way of studying.

For LO2, it is very important to master some English professional terms of engineering, so that we can learn other engineering related courses more efficiently.

For LO3, consolidating the important facets and exercises in the tradition science and research-based technical ability of appropriate reasoning.

For LO4, Let students equip with engineering critical thinking skills have a preparation for further study under the Engineering background. 

	\paragraph {How it helps the major?}

This unit is suitable for students who major in data science because it can assist the study of data science. When students majoring in data science look for errors in experiments, it is easy to leave out details, such as selection bias, observer bias and etc. Besides, data science students should beware of clean data syndrom and not give into the fallacies with the use of critical thinking. Under such circumstances, mastering a skill of engineering critical thinking is essential.

	\paragraph {How it helps the career?}

It is obvious that ENGG1813 Engineering Critical Thinking is useful for our data science students in various senses. For the future career, students majoring in data science can engaged in big data management, research and application development in government agencies, enterprises and companies. At the same time, we can get postgraduate students majoring in software engineering, computer science and technology, and applied statistics or go abroad for further study. Generally, graduates step into those jobs related to the data science major will have more promotion space, and as a compound talent, they will get better development, and their business ability will be better.

	\paragraph {What opportunities it can bring for us in the future?}

In the future, the trend of multidisciplinary crossing is inevitable. Data Science is merely a powerful tool, or a compelling concept. There must be an appropriate object to use it, and engineering is the special one. Demand for big data and accurate measurement is increasing, thereforing learning this unit is bound to obtain work opportunities for data scientists.

	\clearpage

	\subsubsection {ECON 1001}

	\begin{figure}[h]
	\centering
	\includegraphics[width=8cm,height=5cm]{Introductory_Microeconomics}
	\caption{Introductory Microeconomics}
	\end{figure}

	\paragraph {Brief Introduction}

According to the guideline of ECON1001 Introductory Microeconomics, it is a unit of introduction to Microeconomics related to the economic decisions based on individual company and households and how they interact in markets with each other. It is common knowledge that economic problems have frequent impact on Australian society. Introductory Microeconomics lead students to explore the language and analytical framework adopted in Economics for the  social phenomenon and public policy issues. Whatever one's career intentions, coming to grips with economic ideas is essential for understanding society, business and government. Students are given a comprehensive introduction to these ideas and are prepared for the advanced study of microeconomics in subsequent years. Prior knowledge of mathematics is assumed.
	
	\paragraph {Which field it pertains to?}

The elective (ECON1001 Introductory Microeconomics) is from the Faculty of Arts and Social Sciences according to the classification of University of Sydney standard. Briefly, it is under the table s electives- Economy.

	\paragraph {What core knowledge it has?}

By studying this unit, we can master various knowledge mentioned below.
Microeconomics is an economic theory that studies the economic behavior of a single economic unit in a society and how the single value of the corresponding economic variable is determined, and analyzes the economic behavior of an individual economic unit.

	\paragraph {What abilities can it improve?}
	
Learning outcomes are provided here.
	
	\centerline {\textcolor{red}{Learning Outcomes}}
	
	\begin{itemize}
	  \item Application the economic concepts into practice.
	  \item Critically estimate the hypothesis or real issues.
	\end{itemize}

	\centerline {\textcolor{red}{Extra Explanation}}

For LO1, after understanding the equilibrium price theory, consumer behavior theory, producer behavior theory (including production theory, cost theory and market equilibrium theory), distribution theory, general equilibrium theory and welfare economics, market failure and microeconomics policy, we can apply them into our real life problems to deal with decision making.

For LO2, when we meet with problems, with the knowledge of microeconomics, we can take the limitation and benefit both into consideration to find out a solution.

	\paragraph {How it helps the major?}

The knowledge in this unit can be certain useful tool for a students who major in data science because it can assist the study of data science. For example, microeconomics can help data science learner understand consumers' decisions and enterprises make strategic decisions (supply chain) according to consumers' react.

	\paragraph {How it helps the career?}

As is said in the unit outlines of the unit, whatever one's career intentions, coming to grips with economic ideas is essential for understanding society, business and government. 

In terms of future career path, Econ1001 can also be helpful, broadening students’ horizons and providing them with more options and opportunities to select the most suitable jobs. I have mentioned previously that learning microeconomics enables students to see how data science can be utilized for business models and assist precise decision making. Taking microeconomics can also help students to take a peek into the financial and economics disciplines. Considering the amount of data science and data analytics needed in the field of financial market and economics researches, taking microeconomics may be helpful for students to discover a new career interest and have more choices. Data science as an academic discipline is far more than just the knowledge of data. It has great implications in a broad range of career paths, and for students majoring in data science to be successful, they need to know not only data science, but also have a comprehensive understanding in the field that they utilize their data science expertise in. Therefore, microeconomics can be a very sound choice for more opportunities and a broader career path.

	\paragraph {What opportunities it can bring for us in the future?}

An economic course can provide great opportunities about multidisciplinary fields like EC (electronic commerce). Also, Yale University set a combined subject concerning computing and economy in 2019, which means the relationship between these two subjects are no more alienated.

	\clearpage
	
	\subsection{Information Systems}

	\clearpage
	
	\subsection{Software Development}

	\clearpage

	\section{Contributions}

	\clearpage
	

	\bibliography{refs}
	
	
\end{document}
